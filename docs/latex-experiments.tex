\documentclass[]{article}

\usepackage[utf8]{inputenc}

% epigraph for quote at start of section
\usepackage{epigraph}
\setlength\epigraphwidth{8cm}
\setlength\epigraphrule{0pt}
\renewcommand{\epigraphsize}{\small\itshape}

% for hyperlinks and URL links
\usepackage{hyperref}

% Code formatting with the listing package is highly customizable.
\usepackage{listings}
\usepackage{xcolor}
 
\definecolor{codegreen}{rgb}{0,0.6,0}
\definecolor{codegray}{rgb}{0.5,0.5,0.5}
\definecolor{codepurple}{rgb}{0.58,0,0.82}
\definecolor{backcolour}{rgb}{0.95,0.95,0.92}

\lstdefinestyle{python-style}{
    backgroundcolor=\color{backcolour},   
    commentstyle=\color{codegreen},
    keywordstyle=\color{magenta},
    numberstyle=\tiny\color{codegray},
    stringstyle=\color{codepurple},
    basicstyle=\ttfamily\footnotesize,
    breakatwhitespace=false,         
    breaklines=true,                 
    captionpos=b,                    
    keepspaces=true,                 
    numbers=left,                    
    numbersep=5pt,                  
    showspaces=false,                
    showstringspaces=false,
    showtabs=false,                  
    tabsize=2
}

% \usepackage{pgfplots}

\title{\LaTeX{} Experiments}

\author{AeAeA}

\begin{document}

\maketitle

%%%%%%%%%%%%%%%%%%%%%%%%%%%%%%%%%%%%%%%%%%%%%%%%%%%%%%%%%%%
\section{TeX distributions}

\subsection{MacTeX}

The best for Mac.
\begin{verbatim}$ brew cask install mactex\end{verbatim}

\subsection{Visual Studio Code LaTeX Workshop Extension}

LaTeX Workshop is an extension for Visual Studio Code, aiming 
to provide core features for LaTeX typesetting with Visual Studio Code.
\begin{itemize}
    \item \url{https://github.com/James-Yu/LaTeX-Workshop}
    \item \url{https://github.com/James-Yu/LaTeX-Workshop/wiki/Compile}
\end{itemize}
Build LaTeX file by calling the command \verb|Build LaTeX project| from the\\
Command Palette or from the TeX badge. This command is bound to \\
\verb|Cmd+Ctrl+b|

You can change VS Code settings by opening Settings tab:\\ 
\verb|Cmd+, -> Extensions -> LaTeX| \\ 
or, alternatively, by directly editing settings.json file:\\ 
\verb|~/Library/Application\ Support/Code/User/settings.json| \\
Recommended settings for LaTeX Workshop:
\begin{verbatim}
{
    "latex-workshop.view.pdf.viewer": "tab",
    "latex-workshop.latex.outDir": "%DIR%/texout",
    "latex-workshop.latex.autoBuild.run": "never",
    "latex-workshop.latex.autoClean.run": "onBuilt"
}
\end{verbatim}

\subsection{MiKTeX}

Not for Mac. Old MiKTeX installation:\\
\verb+/usr/local/bin/+\\
\verb+/Applications/MiKTeX\ Console.app/+

\subsection{TinyTeX}

TinyTeX - a lightweight, cross-platform, portable, and 
easy-to-maintain \LaTeX{} distribution based on TeX Live. 

Currently TinyTeX works best for R users. Installing and 
maintaining TinyTeX is easy for R users, since the R package 
tinytex has provided wrapper functions.

For other (non-R) users:
\begin{itemize}

    \item See TinyTeX docs \href{https://yihui.org/tinytex/}{here}.

    \item In the directory \\
          \verb|~/Library/TinyTeX/texmf-dist/tex/latex/| \\
          you can find all \LaTeX{} packages installed for TinyTeX.
    
    \item If you compile a LaTeX document and run into an error message 
          like this:\\
          \verb+! LaTeX Error: File `times.sty' not found.+ \\
          It basically indicates a missing LaTeX package.

          Use the command \verb+tlmgr search+ to find the name of 
          the missing package:\\
          \verb+$ tlmgr search --global --file "/times.sty"+\\
          \verb+psnfss: texmf-dist/tex/latex/psnfss/times.sty+

          In this case, the missing package is \verb+psnfss+, and we 
          can install a package via \verb+tlmgr install+, e.g., \\
          \verb+$ tlmgr install psnfss+

          If you still see error messages that you don’t understand, 
          you may need to update everything:\\
          \verb+$ tlmgr update --self --all+\\
          \verb+$ tlmgr path add+\\
          \verb+$ fmtutil-sys --all+
    
    \item To uninstall TinyTeX use command line:\\
          \verb+$ tlmgr path remove+\\
          \verb+$ rm -r "~/Library/TinyTeX"+
          
\end{itemize}


%%%%%%%%%%%%%%%%%%%%%%%%%%%%%%%%%%%%%%%%%%%%%%%%%%%%%%%%%%%
\section{Epigraph}

\epigraph
{In doing what we ought we deserve no praise, because it is our duty.}
{--- \textup{Saint Augustine}}


\subsection{Online docs}
\href{http://texdoc.net/}{TeXdoc Online} is TeX and LaTeX documentation 
lookup system.


\subsection{verbatim}

\begin{verbatim}
Text enclosed inside 
    \begin{verbatim} ... \end {verbatim} 
environment                      is printed directly 
and all \LaTeX{} commands are ignored.
\end{verbatim}

\begin{verbatim*}
Text enclosed inside \begin{verbatim*} environment 
    is printed directly 
and all \LaTeX{} commands are ignored,
and white spaces are emphasized with a special symbol.
\end{verbatim*}

Use \verb| \verb+<inline verbatim text>+ | like this:\\
The \verb+\ldots+ command produces \ldots


\subsection{listings: Source code printing}
\begin{itemize}
    \item \href
    {http://texdoc.net/texmf-dist/doc/latex/listings/listings.pdf}
    {listings} package documentation
    \item \url{https://www.overleaf.com/learn/latex/Code_listing}
\end{itemize}


\subsubsection{minimal setup}

Example of using the \verb+\begin{lstlisting}[language=Python]+ 
environment \\
from the \verb+\usepackage{listings}+ package to highlight Python code:

\begin{lstlisting}[language=Python]
import numpy as np
    
def incmatrix(genl1,genl2):
    m = len(genl1)
    n = len(genl2)
    M = None #to become the incidence matrix
    VT = np.zeros((n*m,1), int)  #dummy variable
    
    #compute the bitwise xor matrix
    M1 = bitxormatrix(genl1)
    M2 = np.triu(bitxormatrix(genl2),1) 
    
    for i in range(m-1):
        for j in range(i+1, m):
            [r,c] = np.where(M2 == M1[i,j])
            for k in range(len(r)):
                VT[(i)*n + r[k]] = 1;
                VT[(i)*n + c[k]] = 1;
                VT[(j)*n + r[k]] = 1;
                VT[(j)*n + c[k]] = 1;
    
                if M is None:
                    M = np.copy(VT)
                else:
                    M = np.concatenate((M, VT), 1)
    
                VT = np.zeros((n*m,1), int)
    
    return M
\end{lstlisting}


\subsubsection{with code styles and colours}

You need \verb+\usepackage{xcolor}+ package for the code colouring. \\
Just like in floats (tables and figures), captions can be added to a 
listing for a more clear presentation. This caption can be later used 
in the list of Listings \verb+\lstlistoflistings+.

\lstset{style=python-style}
\begin{lstlisting}[language=Python, caption=Python example]
import numpy as np
    
def incmatrix(genl1,genl2):
    m = len(genl1)
    n = len(genl2)
    M = None #to become the incidence matrix
    VT = np.zeros((n*m,1), int)  #dummy variable
    
    #compute the bitwise xor matrix
    M1 = bitxormatrix(genl1)
    M2 = np.triu(bitxormatrix(genl2),1) 
    
    for i in range(m-1):
        for j in range(i+1, m):
            [r,c] = np.where(M2 == M1[i,j])
            for k in range(len(r)):
                VT[(i)*n + r[k]] = 1;
                VT[(i)*n + c[k]] = 1;
                VT[(j)*n + r[k]] = 1;
                VT[(j)*n + c[k]] = 1;
    
                if M is None:
                    M = np.copy(VT)
                else:
                    M = np.concatenate((M, VT), 1)
    
                VT = np.zeros((n*m,1), int)
    
    return M
\end{lstlisting}


\newpage
%%%%%%%%%%%%%%%%%%%%%%%%%%%%%%%%%%%%%%%%%%%%%%%%%%%%%%%%%%%
\section{Picture environment}

\begin{itemize}
    \item \url{https://en.wikibooks.org/wiki/LaTeX/Picture}
    \item \url{https://www.overleaf.com/learn/latex/Picture_environment}
\end{itemize}

\vspace{5mm}

\setlength{\unitlength}{1cm}
\thicklines
\begin{picture}(10,6)

\put(0,0){\line(0,1){6}}
\put(0,0){\line(1,0){10}}
\put(10,0){\line(0,1){6}}
\put(0,6){\line(1,0){10}}

\put(0,1){\circle{1}}
\put(0,1){\circle*{2}}
\put(0,1){\oval(10,1.4)[r]}
\end{picture}

\vspace{5mm}

The components of the direction vector \verb+(x1,y1)+ 
of the line segment \\
\verb+\line(x1,y1){length}+ are restricted to the integers 
$(-6,-5, ... , 5,6)$ 
and they have to be coprime. 
The figure below illustrates all 25 possible slope values 
in the first quadrant.

\vspace{5mm}

\setlength{\unitlength}{5cm}
\begin{picture}(1,1)
    \put(0,0){\line(0,1){1}}
    \put(0,0){\line(1,0){1}}
    \put(0,0){\line(1,1){1}}
    \put(0,0){\line(1,2){.5}}
    \put(0,0){\line(1,3){.3333}}
    \put(0,0){\line(1,4){.25}}
    \put(0,0){\line(1,5){.2}}
    \put(0,0){\line(1,6){.1667}}
    \put(0,0){\line(2,1){1}}
    \put(0,0){\line(2,3){.6667}}
    \put(0,0){\line(2,5){.4}}
    \put(0,0){\line(3,1){1}}
    \put(0,0){\line(3,2){1}}
    \put(0,0){\line(3,4){.75}}
    \put(0,0){\line(3,5){.6}}
    \put(0,0){\line(4,1){1}}
    \put(0,0){\line(4,3){1}}
    \put(0,0){\line(4,5){.8}}
    \put(0,0){\line(5,1){1}}
    \put(0,0){\line(5,2){1}}
    \put(0,0){\line(5,3){1}}
    \put(0,0){\line(5,4){1}}
    \put(0,0){\line(5,6){.8333}}
    \put(0,0){\line(6,1){1}}
    \put(0,0){\line(6,5){1}}
\end{picture}

\vspace{5mm}

The picture environment only admits diameters up to 
approximately 14mm, and even below this limit, not all 
diameters are possible.

\setlength{\unitlength}{1mm}
\begin{picture}(60, 40)
    \put(20,30){\circle{1}}
    \put(20,30){\circle{2}}
    \put(20,30){\circle{4}}
    \put(20,30){\circle{8}}
    \put(20,30){\circle{16}}
    \put(20,30){\circle{32}}
    \put(40,30){\circle{1}}
    \put(40,30){\circle{2}}
    \put(40,30){\circle{3}}
    \put(40,30){\circle{4}}
    \put(40,30){\circle{5}}
    \put(40,30){\circle{6}}
    \put(40,30){\circle{7}}
    \put(40,30){\circle{8}}
    \put(40,30){\circle{9}}
    \put(40,30){\circle{10}}
    \put(40,30){\circle{11}}
    \put(40,30){\circle{12}}
    \put(40,30){\circle{13}}
    \put(40,30){\circle{14}}
    \put(15,10){\circle*{1}}
    \put(20,10){\circle*{2}}
    \put(25,10){\circle*{3}}
    \put(30,10){\circle*{4}}
    \put(35,10){\circle*{5}}
\end{picture}

\newpage

\setlength{\unitlength}{1cm}
\begin{picture}(11,4)
    \put(0,0){\circle*{0.5}}
    \put(11,0){\circle*{0.5}}
    \put(0,4){\circle*{0.5}}
    \put(11,4){\circle*{0.5}}

\thicklines
    \put(8,3.3){{\footnotesize $3$-simplex}}
    \put(9,3){\circle*{0.1}}
    \put(8.3,2.9){$a_2$}
    \put(8,1){\circle*{0.1}}
    \put(7.7,0.5){$a_0$}
    \put(10,1){\circle*{0.1}}
    \put(9.7,0.5){$a_1$}
    \put(11,1.66){\circle*{0.1}}
    \put(11.1,1.5){$a_3$}
    \put(9,3){\line(3,-2){2}}
    \put(10,1){\line(3,2){1}}
    \put(8,1){\line(1,0){2}}
    \put(8,1){\line(1,2){1}}
    \put(10,1){\line(-1,2){1}}
\end{picture}

\vspace{20mm}

\setlength{\unitlength}{0.20mm}
\begin{picture}(400,250)
    \put(75,10){\line(1,0){130}}
    \put(75,50){\line(1,0){130}}
    \put(75,200){\line(1,0){130}}
    \put(120,200){\vector(0,-1){150}}
    \put(190,200){\vector(0,-1){190}}
    \put(97,120){$\alpha$}
    \put(170,120){$\beta$}
    \put(220,195){upper state}
    \put(220,45){lower state 1}
    \put(220,5){lower state 2}
\end{picture}

\vspace{20mm}

\setlength{\unitlength}{0.8cm}
\begin{picture}(10,5)
\thicklines

% Bézier curves
\qbezier(1,1)(5,5)(9,0.5)
\put(2,1){{Bézier curve}}

\put(1,1){\circle*{0.2}} % start point
\put(5,5){\circle*{0.2}} % control point
\put(9,0.5){\circle*{0.2}} % endpoint

\end{picture}


\newpage

\lstlistoflistings

\end{document}