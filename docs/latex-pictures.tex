\documentclass[]{article}

% for hyperlinks and URL links
\usepackage{hyperref}

\title{\LaTeX{} Experiments\\
Part II: Pictures
}

\author{AeAeA}

\begin{document}

\maketitle

%%%%%%%%%%%%%%%%%%%%%%%%%%%%%%%%%%%%%%%%%%%%%%%%%%%%%%%%%%%
\section{Picture environment}

\begin{itemize}
    \item \url{https://en.wikibooks.org/wiki/LaTeX/Picture}
    \item \url{https://www.overleaf.com/learn/latex/Picture_environment}
\end{itemize}

\vspace{5mm}

\setlength{\unitlength}{1cm}
\thicklines
\begin{picture}(10,6)

\put(0,0){\line(0,1){6}}
\put(0,0){\line(1,0){10}}
\put(10,0){\line(0,1){6}}
\put(0,6){\line(1,0){10}}

\put(0,1){\circle{2}} % max 14mm 
\put(0,1){\oval(10,1.4)[r]}

\put(0,1){\circle*{1}} % max 5mm
\put(0,1){\oval(10,0.5)[r]}
\end{picture}

\newpage

\subsection{line}
The components of the direction vector \verb+(x1,y1)+ 
of the line segment \\
\verb+\line(x1,y1){length}+ are restricted to the integers 
$(-6,-5, ... , 5,6)$ 
and they have to be coprime. 
The figure below illustrates all 25 possible slope values 
in the first quadrant.

\vspace{5mm}

\setlength{\unitlength}{5cm}
\begin{picture}(1,1)
    \put(1,0){\line(0,1){1}} % right vertical
    \put(0,1){\line(1,0){1}} % top horizontal
    \put(0,0){\line(1,1){1.05}} % diagonal
    \put(0,0){\line(1,2){.5}}
    \put(0,0){\line(1,3){.3333}}
    \put(0,0){\line(1,4){.25}}
    \put(0,0){\line(1,5){.2}}
    \put(0,0){\line(1,6){.1667}}
    \put(0,0){\line(2,1){1}}
    \put(0,0){\line(2,3){.6667}}
    \put(0,0){\line(2,5){.4}}
    \put(0,0){\line(3,1){1}}
    \put(0,0){\line(3,2){1}}
    \put(0,0){\line(3,4){.75}}
    \put(0,0){\line(3,5){.6}}
    \put(0,0){\line(4,1){1}}
    \put(0,0){\line(4,3){1}}
    \put(0,0){\line(4,5){.8}}
    \put(0,0){\line(5,1){1}}
    \put(0,0){\line(5,2){1}}
    \put(0,0){\line(5,3){1}}
    \put(0,0){\line(5,4){1}}
    \put(0,0){\line(5,6){.8333}}
    \put(0,0){\line(6,1){1}}
    \put(0,0){\line(6,5){1}}
\end{picture}


\subsection{circle}
The picture environment only admits diameters up to 
approximately 14mm, and even below this limit, not all 
diameters are possible.

\setlength{\unitlength}{1mm}
\begin{picture}(65, 40)
    \put(1,30){\circle{1}}
    \put(3,30){\circle{2}}
    \put(6,30){\circle{3}}
    \put(10,30){\circle{4}}
    \put(15,30){\circle{5}}
    \put(21,30){\circle{6}}
    \put(28,30){\circle{7}}
    \put(36,30){\circle{8}}
    \put(45,30){\circle{9}}
    \put(55,30){\circle{10}}
    \put(55,20){\circle{10}}
    \put(44,20){\circle{11}}
    \put(32,20){\circle{12}}
    \put(19,20){\circle{13}}
    \put(5,20){\circle{14}}
    \put(5,15){\circle{15}} % cannot be > 14mm
    \put(21,5){\circle*{1}}
    \put(23,5){\circle*{2}}
    \put(26,5){\circle*{3}}
    \put(30,5){\circle*{4}}
    \put(35,5){\circle*{5}}
    \put(40,5){\circle*{6}} % cannot be > 5mm
\end{picture}

\newpage

\subsection{vector and qbezier}
\vspace{5mm}

\setlength{\unitlength}{1cm}
\begin{picture}(11,4)
    \put(0,0){\circle*{0.5}}
    \put(11,0){\circle*{0.5}}
    \put(0,4){\circle*{0.5}}
    \put(11,4){\circle*{0.5}}

\thicklines
    \put(8,3.3){{\footnotesize $3$-simplex}}
    \put(9,3){\circle*{0.1}}
    \put(8.3,2.9){$a_2$}
    \put(8,1){\circle*{0.1}}
    \put(7.7,0.5){$a_0$}
    \put(10,1){\circle*{0.1}}
    \put(9.7,0.5){$a_1$}
    \put(11,1.66){\circle*{0.1}}
    \put(11.1,1.5){$a_3$}
    \put(9,3){\line(3,-2){2}}
    \put(10,1){\line(3,2){1}}
    \put(8,1){\line(1,0){2}}
    \put(8,1){\line(1,2){1}}
    \put(10,1){\line(-1,2){1}}
\end{picture}

\vspace{20mm}

\setlength{\unitlength}{0.20mm}
\begin{picture}(400,250)
    \put(75,10){\line(1,0){130}}
    \put(75,50){\line(1,0){130}}
    \put(75,200){\line(1,0){130}}
    \put(120,200){\vector(0,-1){150}}
    \put(190,200){\vector(0,-1){190}}
    \put(97,120){$\alpha$}
    \put(170,120){$\beta$}
    \put(220,195){upper state}
    \put(220,45){lower state 1}
    \put(220,5){lower state 2}
\end{picture}

\vspace{20mm}

\setlength{\unitlength}{0.8cm}
\begin{picture}(10,5)
\thicklines

% Bézier curves
\qbezier(1,3)(7,5)(9,0.5)

\put(1,3){\circle*{0.2}} % start point
\put(1,2.5){{\footnotesize start}}
\put(7,5){\circle*{0.2}} % control point
\put(6.5,4.5){{\footnotesize control}}
\put(9,0.5){\circle*{0.2}} % end point
\put(8,0.5){{\footnotesize end}}

\put(3,1){{Bézier curve}}

\end{picture}

\end{document}