\documentclass[]{article}

\usepackage{layout} % To print the current details plus the layout shape

\usepackage{showframe} % To render a frame marking the margins 

% for hyperlinks and URL links
\usepackage{hyperref}

\title{\LaTeX{} Experiments\\
Part II: Pictures
}

\author{AeAeA}

\begin{document}

\maketitle

%==============================================================================
\section{Picture environment}

\begin{itemize}
    \item \url{https://en.wikibooks.org/wiki/LaTeX/Picture}
    \item \url{https://www.overleaf.com/learn/latex/Picture_environment}
    \item The default value of \verb+\unitlength+ is 1pt.
\end{itemize}
Picture is the standard tool to create figures in \LaTeX. 
This tool is sometimes too restrictive and cumbersome to work with, 
but it's supported by most of the compilers and no extra packages 
are needed. If you need to create complex figures, for more 
suitable and powerful tools see the TikZ and Pgfplots packages.

\vfill

\setlength{\unitlength}{1cm}
\begin{picture}(11,6)
    \thicklines
    \put(0,0){\circle*{0.2}}

    \put(0,0){\line(0,1){6}} % left
    \put(0,0){\line(1,0){11}} % bottom
    \put(11,0){\line(0,1){6}} % right
    \put(0,6){\line(1,0){11}} % top

    \put(0,1){\circle{1.4}} % max 14mm 
    \put(0,1){\oval(10,1.4)[r]}

    \put(0,1){\circle*{1}} % max 5mm
    \put(0,1){\oval(10,0.5)[r]}
\end{picture}

\newpage

%--------------------------------------
\subsection{layout and several examples}

\vspace{5mm}
\setlength{\unitlength}{1pt}
\begin{picture}(345,40)
    \put(0,0){\circle*{5}}
    \put(345,0){\circle*{5}}
    \put(0,40){\circle*{5}}
    \put(345,40){\circle*{5}}

    \put(5,20){\circle{1}}
    \put(7,20){\circle{2}}
    \put(10,20){\circle{3}}
    \put(14,20){\circle{4}}
    \put(19,20){\circle{5}}
    \put(25,20){\circle{6}}
    \put(32,20){\circle{7}}
    \put(40,20){\circle{8}}
    \put(49,20){\circle{9}}
    \put(59,20){\circle{10}}

    \put(70,20){\circle{11}}
    \put(82,20){\circle{12}}
    \put(95,20){\circle{13}}
    \put(109,20){\circle{14}}
    \put(124,20){\circle{15}}
    \put(140,20){\circle{16}}
    \put(157,20){\circle{17}}
    \put(175,20){\circle{18}}
    \put(194,20){\circle{19}}
    \put(214,20){\circle{20}}

    \put(239,20){\circle{25}}
    \put(269,20){\circle{30}}
    \put(304,20){\circle{35}}
    \put(344,20){\circle{40}}

    \put(0,20){\circle{41}} % cannot be > 40
\end{picture}
\vspace{5mm}

Above is the the picture in first paragraph (no indent), below is picture 
in next paragraph (with indent).

\vspace{5mm}
\setlength{\unitlength}{1pt}
\begin{picture}(330,20)
    \put(0,0){\circle*{5}}
    \put(330,0){\circle*{5}}
    \put(0,20){\circle*{5}}
    \put(330,20){\circle*{5}}

    \put(0,3){\line(1,0){330}}
    \put(0,17){\line(1,0){330}}

    \put(5,10){\circle*{1}}
    \put(7,10){\circle*{2}}
    \put(10,10){\circle*{3}}
    \put(14,10){\circle*{4}}
    \put(19,10){\circle*{5}}
    \put(25,10){\circle*{6}}
    \put(32,10){\circle*{7}}
    \put(40,10){\circle*{8}}
    \put(49,10){\circle*{9}}
    \put(59,10){\circle*{10}}

    \put(70,10){\circle*{11}}
    \put(82,10){\circle*{12}}
    \put(95,10){\circle*{13}}
    \put(109,10){\circle*{14}}
    \put(124,10){\circle*{15}} % cannot be > 14pt

    \put(150,10){\circle*{16}}
    \put(165,10){\circle*{17}}
    \put(180,10){\circle*{20}}
\end{picture}
\vspace{5mm}

\vspace{5mm}
\setlength{\unitlength}{1cm}
\begin{picture}(11,4)
    \put(0,0){\circle*{0.2}}
    \put(11,0){\circle*{0.2}}
    \put(0,4){\circle*{0.2}}
    \put(11,4){\circle*{0.2}}

    \thicklines
    \put(8,3.3){{\footnotesize $3$-simplex}}
    \put(9,3){\circle*{0.1}}
    \put(8.3,2.9){$a_2$}
    \put(8,1){\circle*{0.1}}
    \put(7.7,0.5){$a_0$}
    \put(10,1){\circle*{0.1}}
    \put(9.7,0.5){$a_1$}
    \put(11,1.66){\circle*{0.1}}
    \put(11.1,1.5){$a_3$}
    \put(9,3){\line(3,-2){2}}
    \put(10,1){\line(3,2){1}}
    \put(8,1){\line(1,0){2}}
    \put(8,1){\line(1,2){1}}
    \put(10,1){\line(-1,2){1}}
\end{picture}
\vspace{5mm}

\noindent Some text to demonstrate different effect of 
\verb+\noindent+ on text (here) and on picture (above without 
\verb+\noindent+ and below with \verb+\noindent+). 
Compare the four corner dots for pictures above and below.

\vspace{5mm}
\noindent
\setlength{\unitlength}{0.20mm}
\begin{picture}(400,250)
    \put(0,0){\circle*{10}}
    \put(400,0){\circle*{10}}
    \put(0,250){\circle*{10}}
    \put(400,250){\circle*{10}}

    \put(75,10){\line(1,0){130}}
    \put(75,50){\line(1,0){130}}
    \put(75,200){\line(1,0){130}}
    \put(120,200){\vector(0,-1){150}}
    \put(190,200){\vector(0,-1){190}}
    \put(97,120){$\alpha$}
    \put(170,120){$\beta$}
    \put(220,195){upper state}
    \put(220,45){lower state 1}
    \put(220,5){lower state 2}
\end{picture}
\vspace{5mm}

Regular paragraph (without \verb+\noindent+).

\vspace{5mm}
\setlength{\unitlength}{0.8cm}
\begin{picture}(10,5)
    \put(0,0){\circle*{0.3}}
    \put(10,0){\circle*{0.3}}
    \put(0,5){\circle*{0.3}}
    \put(10,5){\circle*{0.3}}

    \thicklines

    % Bézier curves
    \qbezier(1,3)(7,5)(9,0.5)

    \put(1,3){\circle*{0.2}} % start point
    \put(1,2.5){{\footnotesize start}}
    \put(7,5){\circle*{0.2}} % control point
    \put(6.5,4.5){{\footnotesize control}}
    \put(9,0.5){\circle*{0.2}} % end point
    \put(8,0.5){{\footnotesize end}}

    \put(3,1){{Bézier curve}}

\end{picture}

\newpage

%--------------------------------------
\subsection{line}
\begin{verbatim}\put(x,y){\line(x1,y1){length}}\end{verbatim}
The components of the direction vector \verb+(x1,y1)+ 
of the line segment \\
\verb+\line(x1,y1){length}+ are restricted to the integers 
$(-6,-5, ... , 5,6)$ 
and they have to be coprime. 
The figure below illustrates all 25 possible slope values 
in the first quadrant.

\vspace{5mm}
\setlength{\unitlength}{5cm}
\begin{picture}(1,1)
    \put(0,0){\circle*{0.02}}
    \put(1,0){\circle*{0.02}}
    \put(0,1){\circle*{0.02}}
    \put(1,1){\circle*{0.02}}

    \put(1,0){\line(0,1){1}} % right vertical
    \put(0,1){\line(1,0){3}} % top horizontal
    \put(0,0){\line(1,1){3}} % diagonal
    \put(0,0){\line(1,2){.5}}
    \put(0,0){\line(1,3){.3333}}
    \put(0,0){\line(1,4){.25}}
    \put(0,0){\line(1,5){.2}}
    \put(0,0){\line(1,6){.1667}}
    \put(0,0){\line(2,1){1}}
    \put(0,0){\line(2,3){.6667}}
    \put(0,0){\line(2,5){.4}}
    \put(0,0){\line(3,1){1}}
    \put(0,0){\line(3,2){1}}
    \put(0,0){\line(3,4){.75}}
    \put(0,0){\line(3,5){.6}}
    \put(0,0){\line(4,1){1}}
    \put(0,0){\line(4,3){1}}
    \put(0,0){\line(4,5){.8}}
    \put(0,0){\line(5,1){1}}
    \put(0,0){\line(5,2){1}}
    \put(0,0){\line(5,3){1}}
    \put(0,0){\line(5,4){1}}
    \put(0,0){\line(5,6){.8333}}
    \put(0,0){\line(6,1){1}}
    \put(0,0){\line(6,5){1}}
\end{picture}
\vspace{5mm}

%--------------------------------------
\subsection{vector}
\begin{verbatim}\put(x,y){\vector(x1,y1){length}}\end{verbatim}
For vectors, the components of the direction vector are even more 
narrowly restricted than for line segments, namely to the integers 
$(-4,-3, ... , 3,4)$. Components also have to be coprime. 
Notice the effect of the \verb+\thicklines+ command on the two arrows 
pointing to the upper left.

\vspace{5mm}
\setlength{\unitlength}{0.75mm}
\begin{picture}(60,40)
    \put(0,0){\circle*{1}}
    \put(60,0){\circle*{1}}
    \put(0,40){\circle*{1}}
    \put(60,40){\circle*{1}}

\thinlines
    \put(30,20){\vector(0,1){257}} % vertical up to the border
    \put(30,20){\vector(1,0){187}} % horizontal right to the border

    % all possible angles between 0 and 45 degrees:
    \put(30,20){\vector(4,1){20}}
    \put(30,20){\vector(3,1){20}}
    \put(30,20){\vector(2,1){20}}
    \put(30,20){\vector(3,2){20}}
    \put(30,20){\vector(4,3){20}}
    \put(30,20){\vector(1,1){20}}

\thicklines
    \put(30,20){\vector(-4,1){30}}
    \put(30,20){\vector(-1,4){5}}
\thinlines
    \put(30,20){\vector(-1,-1){100}} % goes to the page border
\end{picture}
\vspace{5mm}

%--------------------------------------
\subsection{circle}
\begin{verbatim}\put(x,y){\circle{diameter}}\end{verbatim}
The picture environment only admits diameters up to approximately 14mm 
(40pt) for circles and 5mm (14pt) for disks, and even below this limit, 
not all diameters are possible.

\vspace{5mm}
\setlength{\unitlength}{1mm}
\begin{picture}(65, 40)
    \put(0,0){\circle*{1}}
    \put(65,0){\circle*{1}}
    \put(0,40){\circle*{1}}
    \put(65,40){\circle*{1}}

    \put(1,30){\circle{1}}
    \put(3,30){\circle{2}}
    \put(6,30){\circle{3}}
    \put(10,30){\circle{4}}
    \put(15,30){\circle{5}}
    \put(21,30){\circle{6}}
    \put(28,30){\circle{7}}
    \put(36,30){\circle{8}}
    \put(45,30){\circle{9}}
    \put(55,30){\circle{10}}
    \put(55,20){\circle{10}}
    \put(44,20){\circle{11}}
    \put(32,20){\circle{12}}
    \put(19,20){\circle{13}}
    \put(5,20){\circle{14}}
    \put(5,15){\circle{15}} % cannot be > 14mm
    \put(21,5){\circle*{1}}
    \put(23,5){\circle*{2}}
    \put(26,5){\circle*{3}}
    \put(30,5){\circle*{4}}
    \put(35,5){\circle*{5}}
    \put(40,5){\circle*{6}} % cannot be > 5mm
\end{picture}
\vspace{5mm}

%--------------------------------------
\subsection{Text and formulae}

\vspace{5mm}
\setlength{\unitlength}{0.8cm}
\begin{picture}(6,5)
    \put(0,0){\circle*{0.1}}
    \put(6,0){\circle*{0.1}}
    \put(0,5){\circle*{0.1}}
    \put(6,5){\circle*{0.1}}

\thicklines
    \put(1,0.5){\line(2,1){3}}
    \put(4,2){\line(-2,1){2}}
    \put(2,3){\line(-2,-5){1}}
    \put(0.7,0.3){$A$}
    \put(4.05,1.9){$B$}
    \put(1.7,2.95){$C$}
    \put(3.1,2.5){$a$}
    \put(1.3,1.7){$b$}
    \put(2.5,1.05){$c$}
    \put(0.3,4){$F=\sqrt{s(s-a)(s-b)(s-c)}$}
    \put(3.5,0.4){$\displaystyle s:=\frac{a+b+c}{2}$}
\end{picture}
\vspace{5mm}

%--------------------------------------
\subsection{multiput}
\begin{verbatim}\multiput(x,y)(dx,dy){n}{object}\end{verbatim}

\vspace{5mm}
\setlength{\unitlength}{2mm}
\begin{picture}(30,20)
    \put(0,0){\circle*{0.5}}
    \put(30,0){\circle*{0.5}}
    \put(0,20){\circle*{0.5}}
    \put(30,20){\circle*{0.5}}

\linethickness{0.5pt}
    \multiput(0,0)(1,0){31}{\line(0,1){20}}
    \multiput(0,0)(0,1){21}{\line(1,0){25}}

\linethickness{1pt}
    \multiput(0,0)(5,0){6}{\line(0,1){20}}
    \multiput(0,0)(0,5){5}{\line(1,0){25}}
\end{picture}
\vspace{5mm}

%--------------------------------------
\subsection{oval}
\begin{verbatim}\put(x,y){\oval(w,h)[position:b,t,l,r]}\end{verbatim}

\vspace{5mm}
\setlength{\unitlength}{0.75cm}
\begin{picture}(6,4)
    \put(0,0){\circle*{0.2}}
    \put(6,0){\circle*{0.2}}
    \put(0,4){\circle*{0.2}}
    \put(6,4){\circle*{0.2}}

\linethickness{0.5pt}
    \multiput(0,0)(1,0){7}{\line(0,1){4}}
    \multiput(0,0)(0,1){5}{\line(1,0){6}}

\thicklines
    \put(2,3){\oval(3,1.8)}
\thinlines
    \put(3,2){\oval(3,1.8)}
\thicklines
    \put(2,1){\oval(3,1.8)[tl]}
    \put(4,1){\oval(3,1.8)[b]}
    \put(4,3){\oval(3,1.8)[r]}
    \put(3,1.5){\oval(1.8,0.4)}
\end{picture}
\vspace{5mm}

%--------------------------------------
\subsection{Predefined picture boxes}
\begin{verbatim}
\newsavebox{name}
\savebox{name}(width,height)[position]{content}
\put(x,y){\usebox{name}}
\end{verbatim}

\vspace{5mm}
\setlength{\unitlength}{0.5mm}
\begin{picture}(120,168)
    \put(0,0){\circle*{2}}
    \put(120,0){\circle*{2}}
    \put(0,168){\circle*{2}}
    \put(120,168){\circle*{2}}

\newsavebox{\foldera}
\savebox{\foldera}(40,32)[bl]{% definition
    \multiput(0,0)(0,28){2}{\line(1,0){40}}
    \multiput(0,0)(40,0){2}{\line(0,1){28}}
    \put(1,28){\oval(2,2)[tl]}
    \put(1,29){\line(1,0){5}}
    \put(9,29){\oval(6,6)[tl]}
    \put(9,32){\line(1,0){8}}
    \put(17,29){\oval(6,6)[tr]}
    \put(20,29){\line(1,0){19}}
    \put(39,28){\oval(2,2)[tr]}
}

\newsavebox{\folderb}
\savebox{\folderb}(40,32)[l]{% definition
    \put(0,14){\line(1,0){8}}
    \put(8,0){\usebox{\foldera}}
}

    \put(34,26){\line(0,1){102}}
    \put(14,128){\usebox{\foldera}}
    \multiput(34,86)(0,-37){3}{\usebox{\folderb}}

\end{picture}

%--------------------------------------
\subsection{Quadratic Bézier curves}
\begin{verbatim}\qbezier(x1,y1)(x,y)(x2,y2)\end{verbatim}

\newpage
\setlength{\unitlength}{1pt}
\begin{flushleft}
\begin{picture}(345,550)(-10,-10)

% border corners
    \put(-10,-10){\circle*{5}}
    \put(335,-10){\circle*{5}}
    \put(-10,540){\circle*{5}}
    \put(335,540){\circle*{5}}

% grid
\linethickness{0.25pt}
    \multiput(0,0)(10,0){34}{\line(0,1){530}}
    \multiput(0,0)(0,10){54}{\line(1,0){330}}
\linethickness{0.5pt}
    \multiput(0,0)(50,0){7}{\line(0,1){530}}
    \multiput(0,0)(0,50){11}{\line(1,0){330}}
\linethickness{1pt}
    \multiput(0,0)(100,0){4}{\line(0,1){530}}
    \multiput(0,0)(0,100){6}{\line(1,0){330}}

    \put(-30,0){{\footnotesize $0$}}
    \put(-30,100){{\footnotesize $100$}}
    \put(-30,200){{\footnotesize $200$}}
    \put(-30,300){{\footnotesize $300$}}
    \put(-30,400){{\footnotesize $400$}}
    \put(-30,500){{\footnotesize $500$}}
    \put(-30,600){{\footnotesize $600$}}

    \put(0,-20){{\footnotesize $0$}}
    \put(100,-20){{\footnotesize $100$}}
    \put(200,-20){{\footnotesize $200$}}
    \put(300,-20){{\footnotesize $300$}}
    \put(400,-20){{\footnotesize $400$}}

% qbezier examples first column

\linethickness{2pt}
    \qbezier(0,400)(10,450)(0,500)
    \thinlines
    \put(0,400){\line(1,5){10}}
    \put(10,450){\circle*{5}}
    \put(0,500){\line(1,-5){10}}

\linethickness{2pt}
    \qbezier(0,400)(50,450)(0,500)
    \thinlines
    \put(0,400){\line(1,1){50}}
    \put(50,450){\circle*{5}}
    \put(0,500){\line(1,-1){50}}

\linethickness{2pt}
    \qbezier(50,400)(150,450)(50,500)
    \thinlines
    \put(50,400){\line(2,1){100}}
    \put(150,450){\circle*{5}}
    \put(50,500){\line(2,-1){100}}

\linethickness{2pt}
    \qbezier(0,300)(20,300)(0,400)
    \thinlines
    \put(0,300){\line(1,0){20}}
    \put(20,300){\circle*{5}}
    \put(0,400){\line(1,-5){20}}

\linethickness{2pt}
    \qbezier(0,200)(50,150)(0,300)
    \thinlines
    \put(0,200){\line(1,-1){50}}
    \put(50,150){\circle*{5}}
    \put(0,300){\line(1,-3){50}}

\linethickness{2pt}
    \qbezier(0,100)(50,0)(0,200)
    \thinlines
    \put(0,100){\line(1,-2){50}}
    \put(50,0){\circle*{5}}
    \put(0,200){\line(1,-4){50}}

\linethickness{2pt}
    \qbezier(0,0)(100,600)(0,100)
    \thinlines
    \put(0,0){\line(1,6){100}}
    \put(100,600){\circle*{5}}
    \put(0,100){\line(1,5){100}}

% qbezier examples second column

\linethickness{2pt}
    \qbezier(100,400)(400,450)(100,500)
    \thinlines
    \put(100,400){\line(6,1){300}}
    \put(400,450){\circle*{5}}
    \put(100,500){\line(6,-1){300}}

\linethickness{2pt}
    \qbezier(100,300)(400,300)(100,400)
    \thinlines
    \put(100,300){\line(1,0){300}}
    \put(400,300){\circle*{5}}
    \put(100,400){\line(3,-1){300}}

\linethickness{2pt}
    \qbezier(100,200)(400,150)(100,300)
    \thinlines
    \put(100,200){\line(6,-1){300}}
    \put(400,150){\circle*{5}}
    \put(100,300){\line(2,-1){300}}

\linethickness{2pt}
    \qbezier(100,100)(400,0)(100,200)
    \thinlines
    \put(100,100){\line(3,-1){300}}
    \put(400,0){\circle*{5}}
    \put(100,200){\line(3,-2){300}}

\linethickness{2pt}
    \qbezier(100,0)(300,600)(100,100)
    \thinlines
    \put(100,0){\line(1,3){200}}
    \put(300,600){\circle*{5}}
    \put(100,100){\line(2,5){200}}

\end{picture}
\end{flushleft}

\end{document}